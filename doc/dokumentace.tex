% @file dokumentace.tex
% @author Štěpán Faragula
% @brief Dokumentace semestrální práce.
% @version 1.0
% @date 2022-29-12

% Document
\documentclass[12pt]{report}

% Čeština
\usepackage[utf8]{inputenc}
\usepackage[IL2]{fontenc}
\usepackage[czech]{babel}
\usepackage{fontspec}

% Formát dokumentu
\usepackage{caption}
\usepackage{graphicx}
\usepackage{textcomp}
\usepackage{xspace}
\usepackage{parskip}
\graphicspath{{img/}}
\usepackage[
	left=30mm, 
	right=30mm, 
	top=30mm, 
	bottom=30mm,
]{geometry}

% Vychytávky
\usepackage{lipsum}			% Lorem ipsum
\usepackage{menukeys}		% Klavesy
\usepackage{dirtree}		% Adresarova struktura

% Macra
\newcommand\la{\textlangle}  					% levá závorka <
\newcommand\ra{\textrangle}						% pravá závorka >
\newcommand\laratexttt[1]{\la\texttt{#1}\ra}	% texttt v závorkách <>
\newcommand\laratextit[1]{\la\textit{#1}\ra}	% textit v závorkách <>
\newcommand\indentt[1]{						
	\setlength\parindent{5mm}
	#1
	\setlength\parindent{0mm}
	}											% odstavec

% Begin
\begin{document}
	
	% Titulní strana
	\begin{titlepage}
		\centering
		\Large
		
		\includegraphics[width=.7\textwidth]{fav}
		
		\vspace{15mm}
		{\Huge\bfseries Identifikace spamu naivním bayesovským klasifikátorem}
		
		\vspace{15mm}
		{\LARGE Semestrální práce KIV/PC}
		
		\vfill
		\raggedright
		Štěpán Faragula\\
		A21B0119P
		\hfill 
		\today
	\end{titlepage}
	
	% Obsah
	\tableofcontents

	% Zadání
	\chapter{Zadání}
	Při volbě zadání semestrální práce jsme měli na výběr z následujících možností:
		
	\begin{enumerate}
		\item Hledání kořenů rovnice
		\item Identifikace spamu naivním bayesovským klasifikátorem
		\item Celočíselná kalkulačka s neomezenou přesností
	\end{enumerate}
		
	V této práci je popsáno řešení práce \textbf{číslo 2}.
	
	\section{Detaily zadání}
	Naprogramujte v ANSI C přenositelnou \textbf{konzolovou aplikaci}, která bude \textbf{rozhodovat, zda úsek textu} (textový soubor předaný jako parametr na příkazové řádce) \textbf{je nebo není spam}.
	
	Program bude přijímat z příkazové řádky celkem \textbf{sedm} parametrů: První dva parametry budou vzor jména a počet trénovacích souborů obsahujících nevyžádané zprávy (tzv. \textbf{spam}). Třetí a čtvrtý parametr budou vzor jména a počet trénovacích souborů obsahujících vyžádané zprávy (tzv. \textbf{ham}). Pátý a šestý parametr budou vzor jména a počet testovacích souborů. Sedmý para- metr představuje jméno výstupního textového souboru, který bude po dokončení činnosti Vašeho	programu obsahovat výsledky klasifikace testovacích souborů.
	
	Program se tedy bude spouštět příkazem
	
	\indentt{\texttt{spamid.exe} \laratexttt{spam} \laratexttt{spam-cnt} \laratexttt{ham} \laratexttt{ham-cnt} \laratexttt{test} \laratexttt{test-cnt} \laratexttt{out-file}$\,$\keys{\return}}
	
	Symboly \laratexttt{spam}, \laratexttt{ham} a \laratexttt{test} představují vzory jména vstupních souborů. Symboly \laratexttt{spam-cnt}, \laratexttt{ham-cnt} a \laratexttt{test-cnt} představují počty vstupních souborů. Vstupní soubory mají následující pojmenování: \texttt{vzorN}, kde \texttt{N} je celé číslo z intervalu \la1;\textit{N}\ra. Přípona všech vstupních souborů je \texttt{.txt}, přípona není součástí vzoru. Váš program tedy může být během testování spuštěn například takto:
	
	\indentt{
	\texttt{spamid.exe spam 10 ham 20 test 50 result.txt}$\,$\keys{\return}
	}
	
	Výsledkem činnosti programu bude textový soubor, který bude obsahovat seznam testovaných souborů a jejich klasifikaci (tedy rozhodnutí, zda je o spam či neškodný obsah – ham).
	
	Pokud nebude na příkazové řádce uvedeno právě sedm argumentů, vypište chybové hlášení a stručný návod k použití programu v angličtině podle běžných zvyklostí (viz např. ukázková semestrální práce na webu předmětu Programování v jazyce C). \textbf{Vstupem programu jsou pouze argumenty na příkazové řádce – interakce s uživatelem pomocí klávesnice či myši v průběhu práce programu se neočekává.}
	
	Hotovou práci odevzdejte v jediném archivu typu ZIP prostřednictvím automatického odevzdávacího a validačního systému. Postupujte podle instrukcí uvedených na webu předmětu. Archiv nechť obsahuje všechny zdrojové soubory potřebné k přeložení programu, \textbf{makefile} pro Windows i Linux (pro překlad v Linuxu připravte soubor pojmenovaný \texttt{makefile} a pro Windows \texttt{makefile.win}) a dokumentaci ve formátu PDF vytvořenou v typografickém systému \TeX, resp. \LaTeX. Bude-li některá z částí chybět, kontrolní skript Vaši práci odmítne.

	% Analýza úlohy
	\chapter{Analýza úlohy}
	V úloze máme za úkol \textbf{zařadit soubory} do jedné ze dvou tříd – \textbf{spam} či \textbf{ham}. Je nám výrazně doporučeno použít \textbf{naivní bayesovský klasifikátor}.
	
	\section{Definice problému}
	Testovací soubory budeme klasifikovat podle jejich obsažených slov. Budeme tedy pro každý soubor potřebovat vytvořit jakýsi slovník. Tento slovník bude uchovávat informaci o každém 
	
	
	Nejprve potřebujeme vytvořit slovník 
	Nejprve tedy vytvoříme \textbf{slovník klasifikátoru} načtením všech trénovacích souborů. U každého slova budeme uchovávat jeho \textbf{počet} a \textbf{pravděpodobnost výskytu}. 
	
	Testovací soubory budeme třídit podle \textbf{slov}, která obsahuje. Z těchto slov vytvoříme \textbf{slovník}, který budeme porovnávat se \textbf{slovníkem klasifikátoru}. V zadání nám je doporučeno použít \textbf{naivní bayesovský klasifikátor}.
	
	K vyřešení úlohy se potřebujeme vyřešit \textbf{následující problémy}:
	\begin{itemize}
		\item 
		\item Volba vhodné struktury pro slovník slov
		\item Samotná klasifikace testovacích souborů
	\end{itemize}
	
	\section{Zvolení datové struktury slovníku}
	Slovník musí obsahovat záznam o každém slově, které se načetlo.
	Potřebujeme využít takovou strukturu, která nám umožní rychle vyhledávat slova v obsáhlém slovníku. 
	
	Jelikož ANSI C nenabízí žádnou takovou strukturu, budeme si ji muset naprogramovat sami.
	
	% Popis implementace
	\chapter{Popis implementace}
	Příliš žluťoučký kůň úpěl ďábelské ódy.
	
	% Uživatelská příručka
	\chapter{Uživatelská příručka}
	Příliš žluťoučký kůň úpěl ďábelské ódy.
	
	% Závěr
	\chapter{Závěr}
	Příliš žluťoučký kůň úpěl ďábelské ódy.
	
\end{document}
