% @file dokumentace.tex
% @author Štěpán Faragula
% @brief Dokumentace semestrální práce.
% @version 1.0
% @date 2022-29-12

% Document
\documentclass[12pt]{report}

% Čeština
\usepackage[utf8]{inputenc}
\usepackage[IL2]{fontenc}
\usepackage[czech]{babel}
\usepackage{fontspec}

% Formát dokumentu
\usepackage{amsmath}
\usepackage{caption}
\usepackage{graphicx}
\usepackage{textcomp}
\usepackage{xspace}
\usepackage{parskip}
\graphicspath{{img/}}
\usepackage[
	left=30mm, 
	right=30mm, 
	top=30mm, 
	bottom=30mm,
]{geometry}

% Vychytávky
\usepackage{lipsum}			% Lorem ipsum
\usepackage{menukeys}		% Klavesy
\usepackage{dirtree}		% Adresarova struktura

% Macra
\newcommand\la{\textlangle}  					% levá závorka <
\newcommand\ra{\textrangle}						% pravá závorka >
\newcommand\laratexttt[1]{\la\texttt{#1}\ra}	% texttt v závorkách <>
\newcommand\laratextit[1]{\la\textit{#1}\ra}	% textit v závorkách <>
\newcommand\indentt[1]{						
	\setlength\parindent{5mm}
	#1
	\setlength\parindent{0mm}
	}											% odstavec

% Begin
\begin{document}
	
	% Titulní strana
	\begin{titlepage}
		\centering
		\Large
		
		\includegraphics[width=.7\textwidth]{fav}
		
		\vspace{15mm}
		{\Huge\bfseries Identifikace spamu naivním bayesovským klasifikátorem}
		
		\vspace{15mm}
		{\LARGE Semestrální práce KIV/PC}
		
		\vfill
		\raggedright
		Štěpán Faragula\\
		A21B0119P
		\hfill 
		\today
	\end{titlepage}
	
	% Obsah
	\tableofcontents

	% Zadání
	\chapter{Zadání}
	Při volbě zadání semestrální práce jsme měli na výběr z následujících možností:
		
	\begin{enumerate}
		\item Hledání kořenů rovnice
		\item Identifikace spamu naivním bayesovským klasifikátorem
		\item Celočíselná kalkulačka s neomezenou přesností
	\end{enumerate}
		
	V této práci je popsáno řešení práce \textbf{číslo 2}.
	
	\section{Detaily zadání}
	Naprogramujte v ANSI C přenositelnou \textbf{konzolovou aplikaci}, která bude \textbf{rozhodovat, zda úsek textu} (textový soubor předaný jako parametr na příkazové řádce) \textbf{je nebo není spam}.
	
	Program bude přijímat z příkazové řádky celkem \textbf{sedm} parametrů: První dva parametry budou vzor jména a počet trénovacích souborů obsahujících nevyžádané zprávy (tzv. \textbf{spam}). Třetí a čtvrtý parametr budou vzor jména a počet trénovacích souborů obsahujících vyžádané zprávy (tzv. \textbf{ham}). Pátý a šestý parametr budou vzor jména a počet testovacích souborů. Sedmý para- metr představuje jméno výstupního textového souboru, který bude po dokončení činnosti Vašeho	programu obsahovat výsledky klasifikace testovacích souborů.
	
	Program se tedy bude spouštět příkazem
	
	\indentt{\texttt{spamid.exe} \laratexttt{spam} \laratexttt{spam-cnt} \laratexttt{ham} \laratexttt{ham-cnt} \laratexttt{test} \laratexttt{test-cnt} \laratexttt{out-file}$\,$\keys{\return}}
	
	Symboly \laratexttt{spam}, \laratexttt{ham} a \laratexttt{test} představují vzory jména vstupních souborů. Symboly \laratexttt{spam-cnt}, \laratexttt{ham-cnt} a \laratexttt{test-cnt} představují počty vstupních souborů. Vstupní soubory mají následující pojmenování: \texttt{vzorN}, kde \texttt{N} je celé číslo z intervalu \la1;\textit{N}\ra. Přípona všech vstupních souborů je \texttt{.txt}, přípona není součástí vzoru. Váš program tedy může být během testování spuštěn například takto:
	
	\indentt{
	\texttt{spamid.exe spam 10 ham 20 test 50 result.txt}$\,$\keys{\return}
	}
	
	Výsledkem činnosti programu bude textový soubor, který bude obsahovat seznam testovaných souborů a jejich klasifikaci (tedy rozhodnutí, zda je o spam či neškodný obsah – ham).
	
	Pokud nebude na příkazové řádce uvedeno právě sedm argumentů, vypište chybové hlášení a stručný návod k použití programu v angličtině podle běžných zvyklostí (viz např. ukázková semestrální práce na webu předmětu Programování v jazyce C). \textbf{Vstupem programu jsou pouze argumenty na příkazové řádce – interakce s uživatelem pomocí klávesnice či myši v průběhu práce programu se neočekává.}
	
	Hotovou práci odevzdejte v jediném archivu typu ZIP prostřednictvím automatického odevzdávacího a validačního systému. Postupujte podle instrukcí uvedených na webu předmětu. Archiv nechť obsahuje všechny zdrojové soubory potřebné k přeložení programu, \textbf{makefile} pro Windows i Linux (pro překlad v Linuxu připravte soubor pojmenovaný \texttt{makefile} a pro Windows \texttt{makefile.win}) a dokumentaci ve formátu PDF vytvořenou v typografickém systému \TeX, resp. \LaTeX. Bude-li některá z částí chybět, kontrolní skript Vaši práci odmítne.

	% Analýza úlohy
	\chapter{Analýza úlohy}
	V úloze máme za úkol \textbf{zařadit soubory} do jedné ze dvou tříd – \textbf{spam} či \textbf{ham}. Je nám výrazně doporučeno použít \textbf{naivní bayesovský klasifikátor}, není tedy důvod rozebírat, jaký způsob klasifikace zvolíme.
	
	\section{Naivní bayesovský klasifikátor}
	Algoritmus, který má dvě fáze – \textbf{fáze učení} a \textbf{fáze klasifikace}.
	
		\subsection{Fáze učení}
		V této fázi vycházíme z předpokladu, že máme k dispozici \textbf{trénovací soubory} obsahující pouze slova označena jako spam nebo ham. Každý soubor přečteme a vytvoříme tak \textbf{slovník klasifikátoru}. U každého slova budeme uchovávat informaci o jeho \textbf{počtu výskytu} v trénovacích souborech a jeho \textbf{podmíněnou pravděpodobnost výskytu}. Takto vytvořený slovník by měl ještě obsahovat \textbf{apriorní pravděpodobnost}, která analýzou vstupních souborů stanoví \textbf{výchozí pravděpodobnost výskytu spamu či hamu}.
	
		\subsection{Fáze klasifikace}
		Testovací soubory budeme klasifikovat podle obsažených slov. Průběh klasifikace je takový, že každému slovu vyskytující se v \textbf{testovacím souboru a zároveň ve slovníku klasifikátoru} přiřadíme \textbf{podmíněnou pravděpodobnosti výskytu} slova ve spamu a hamu. Dále sečteme logaritmy všech \textbf{podmíněných pravděpodobností} slov a přičteme logaritmus \textbf{apriorní pravděpodobnosti} třídy. Soubor zařadíme do té třídy, která bude mít \textbf{největší výsledek}. Klasifikace souboru je popsána rovnicí \ref{equ:megaVzorec}
		
		\begin{equation}
			c = \arg\max_{c_i \in C}\left( \log (P(c_i)) + \sum_{k\,\in\,\text{\bfseries pozice}} \log (P(\la \text{word}_k | c_i \ra))\right)
			\label{equ:megaVzorec}
		\end{equation}
	
		kde:
		\begin{itemize}
			\item $C$ - množina všech tříd
			\item $c_i$ - označení třídy (spam, ham)
			\item $P(c_i)$ - apriorní pravděpodobnost dané třídy
			\item $P(\la \text{word}_k | c_i \ra)$ - podmíněná pravděpodobnost výskytu slova 
		\end{itemize}
			
	\section{Definice problému}
	Klasifikátor potřebuje ke své činnosti \textbf{slovník}. \textbf{Slovník} můžeme vnímat jako \textbf{datovou strukturu}, která dokáže vrátit výsledek podle hledaného klíče (slova). Chceme tedy, aby tato struktura vkládala a vyhledávala slova \textbf{co nejrychleji} pro obsáhlý slovník. Jelikož úlohu vytváříme v ANSI C, musíme si tuto strukturu naprogramovat sami. 
	
	Dále potřebujeme vyřešit \textbf{způsob klasifikace souborů}.
	
	\section{Volba datové struktury pro slovník}
	Slovník bude obsahovat informaci o všech \textbf{unikátních slov} ze souboru. Ve slovníku nepotřebujeme mazat položky, stačí nám pouze funkce \textbf{přidání} a \textbf{hledání}. Chceme tedy, aby tyto operace pracovali co nejrychleji.
	
		\subsection{Dynamické pole}
	
		\subsection{Hash tabulka}
		
		\subsection{Trie}
	
	\section{Způsob klasifikace}
	Zde se nabízí dvě možnosti: (i) Vytvoříme ještě jeden \textbf{slovník} obsahující slova testovacího souboru, který následně \textbf{porovnáme se slovníkem klasifikátoru}, (ii) nebo jednotlivá slova budeme porovnávat tzv. \textit{on the fly} způsobem, kde \textbf{postupným čtením slov} rovnou klasifikujeme soubor jako spam či ham \textbf{bez vytváření dalšího slovníku}. Výhoda přístupu (i) je uchování čitelnosti programu za cenu vyšší paměťové náročnosti. Výhoda přístupu (ii) je jednoduchost implementace bez nároků na paměť.
	
	% Popis implementace
	\chapter{Popis implementace}
	Příliš žluťoučký kůň úpěl ďábelské ódy.
	
	% Uživatelská příručka
	\chapter{Uživatelská příručka}
	Příliš žluťoučký kůň úpěl ďábelské ódy.
	
	% Závěr
	\chapter{Závěr}
	Příliš žluťoučký kůň úpěl ďábelské ódy.
	
\end{document}
