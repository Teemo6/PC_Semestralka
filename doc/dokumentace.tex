% @file dokumentace.tex
% @author Štěpán Faragula
% @brief Dokumentace semestrální práce.
% @version 1.0
% @date 2022-29-12

% Document class
\documentclass[12pt]{report}

% Use package
\usepackage[utf8]{inputenc}
\usepackage[IL2]{fontenc}
\usepackage[czech]{babel}
\usepackage{amsmath}
\usepackage[hidelinks]{hyperref}
\usepackage[nottoc]{tocbibind}
\usepackage{graphicx}
\graphicspath{{img/}}

\usepackage[
left=30mm, 
right=30mm, 
top=30mm, 
bottom=30mm,
]{geometry}

% Begin document
\begin{document}
	
	% Titulní strana
	\begin{titlepage}
		\centering
		\Large
		\sffamily
		
		\includegraphics[width=.7\textwidth]{fav}
		
		Semestrální práce z předmětu
		
		Programování v jazyku C
		
		% Vertikální mezera 15 mm.
		\vspace{20mm}
		{\Huge\bfseries Identifikace spamu naivním bayesovským klasifikátorem}
		
		\vspace{20mm}
		% Čas je získán ze systému.
		\today 
		
		\vfill			% Vyplní prostor
		\raggedright	% Vše bude zarováno do leva.
		Kamil Ekštein\\		% Příkaz \\ provede násilný zlom řádky.
		A95097\\
		\texttt{kekstein@students.zcu.cz}
		
		\vspace{\baselineskip}
		\emph{Vyučující:}\\
		Ing. Kamil Ekštein, Ph.D.\\
		\texttt{kekstein@kiv.zcu.cz}
	\end{titlepage}
	
	% Obsah
	\tableofcontents

	% Zadání
	\chapter{Zadání}
	Příliš žluťoučký kůň úpěl ďábelské ódy.
	
	% Analýza úlohy
	\chapter{Analýza úlohy}
	Příliš žluťoučký kůň úpěl ďábelské ódy.
	
	% Popis implementace
	\chapter{Popis implementace}
	Příliš žluťoučký kůň úpěl ďábelské ódy.
	
	% Uživatelská příručka
	\chapter{Uživatelská příručka}
	Příliš žluťoučký kůň úpěl ďábelské ódy.
	
	% Závěr
	\chapter{Závěr}
	Příliš žluťoučký kůň úpěl ďábelské ódy.
	
\end{document}
